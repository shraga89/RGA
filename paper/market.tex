
%\todo{plot out our vision on data markets?}\\
%\todo{create a figure with an overview of data markets?}\\

Next, we layout our data market setting partially following~\cite{castro2020data}.

\subsection{Data Market Model}\label{sec:market_model}

A \emph{data market} $\market$ is a composition of three sets of entities, namely \emph{buyers} $\buyerset$, \emph{sellers} $\sellerset$, and \emph{mediator} \mediator. We shall refer to the first two entities also as the \emph{players} $\playerset = \buyerset\cup\sellerset$ in the market. Players may be individuals or groups (\emph{e.g.,} companies or organizations) interested in trading datasets. We use $\productset$ to denote the set of authorized datasets in $\market$. Usually, a dataset $\product$ is associated with a domain and several of other properties that represent it including \emph{e.g.}, the features it contains. Each buyer $\buyer\in\buyerset$ is interested in a set of datasets $\product_{\buyer}\subseteq\productset$ whereas each seller offers a set of products $\product_{\seller}\subseteq\productset$. The mediator is in charge of the transactions between the different players in the market.

\roee{say something about the value of data (find references)}\\

In what follows, each player $\player\in\playerset$ has a different utility from a dataset, according to which they can set prices. A market is a temporal ecosystem. As such, we assume that the market has a finite horizon $\horizon$ and each interaction between the different entities takes place in a discrete timestamp $t<T$.
