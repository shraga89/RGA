Focusing on buyers, we identify two main factors that affect buying data in the setting defined in Section~\ref{sec:datamarket}. The first focuses on selecting the set of datasets a buyer is willing to buy at time $t$. The second is pricing in the form of cost estimation and bids. 

\subsection{Datasets Allocation Optimization Strategy}
Next, we define the \emph{datasets allocation} problem with respect to a single player $\player$ before timestamp $t$ in the market horizon $T$.

Recall that a player $\player$ has a $\productset_ = \langle \product_1, \product_2, \dots, \product_n\rangle$ (set of relevant datasets), $\budget$ (budget), and a valuation value $\valuation_i$ for each relevant dataset $\product_i$. At timestamp $0$, the set of datasets $\productset$, its corresponding valuations $\valuationset$ and an initial $\budget$ are set. While the valuation values of datasets stays constant throughout the horizon, the $\productset$ and $\budget$ change with respect to the interaction in the market. We denote the datasets set available at time $t$ and the current budget at time $t$ as $\productset^{t}$ and $\budget^{t}$, respectively. We assume that the only thing that changes the datasets is that a player has purchased a dataset and thus, $\productset^{t} \subseteq \productset^{t^{\prime}}, t^{\prime}<t$ and $\productset^{0} = \productset$. The budget changes with respect to the revenue from each purchased product and its cost.  

%Recall $\productset_ = \langle \product_1, \product_2, \dots, \product_n\rangle$ as the set of relevant products for the player $\player$. Generally, a player has a budget $\budget$ and a valuation value $\valuation_i$ for each relevant product $\product_i$. At timestamp $0$, the set of products $\productset$, its corresponding valuations $\valuationset$ and an initial $\budget$ are set. While the valuation values of products stays constant throughout the horizon, $\productset$ and $\budget$ change with respect to the interaction in the market. We denote the products set available at time $t$ and the current budget at time $t$ as $\productset^{t}$ and $\budget^{t}$, respectively. We assume that the only thing that changes the product set is that a player has purchased the product and thus, $\productset^{t} \subseteq \productset^{t^{\prime}}, t^{\prime}<t$ and $\productset^{0} = \productset$. The budget changes with respect to the revenue from each purchased product and its cost.  

At time $t$ the player has to select a subset of datasets $\bar{\productset} \subseteq \productset^{t-1}$ that maximizes her future revenues (recall Section~\ref{sec:market_model}). To simplify the notation, we denote that number of datasets available at time $t$ as $m\leq n$. In what follows, let $\costset = \langle \cost_1, \cost_2, \dots, \cost_{m}\rangle, c_i\in {\rm I\!R}$ and $\winset = \langle \win_1, \win_2, \dots, \win_m\rangle, \win_{m}\in \{0,1\}$ represent a realization of costs and win indicators of the relevant datasets after timestamp $t$ has completed, respectively.

In practice, a player has to allocate a subset of products at the beginning of time $t$. Thus, the player does not know the actual cost of datasets when the dataset allocation takes place. Accordingly, the player has to estimate the costs $\hat{\costset} = \langle \hat{\cost}_1, \hat{\cost}_2, \dots, \hat{\cost}_m\rangle$ and wining indicators $\hat{\winset} = \langle \hat{\win}_1, \hat{\win}_2, \dots, \hat{\win}_m\rangle$. Using these estimations, the \emph{datasets allocation} problem can be formalized as follows:

\begin{equation}\label{eq:main1}
\begin{aligned}
& \underset{X}{\text{minimize}} & & \sum_{i=1}^{m} (\valuation_{i} - \hat{\cost}_{i})\cdot\hat{\win}_{i}\cdot X_{i} \\
& \text{subject to} & & \sum_{i=1}^{m} \hat{\cost}_{i}\cdot\hat{\win}_{i}\cdot X_{i} \leq \budget \\
& & & X_{i} \in \{0,1\} \; i = 1, \ldots, m. & 
\end{aligned}
\end{equation}

\noindent Recalling that $\hat{\win}_{i}\in \{0,1\}$ the player can actually decrease the size of $\productset^{t-1}$ to the set of product she estimates she would win, \emph{i.e.,} $win(\productset^{t-1}) = \{p_{i}\in \productset^{t-1} | \hat{w}_{i} = 1\}$. We denote the size of $win(\productset^{t-1})$ as $m_{win}$  

\begin{equation}\label{eq:main2}
\begin{aligned}
& \underset{X}{\text{minimize}} & & \sum_{i=1}^{m_{win}} (\valuation_{i} - \hat{\cost}_{i})\cdot X_{i} \\
& \text{subject to} & & \sum_{i=1}^{m_{win}} \hat{\cost}_{i}\cdot X_{i} \leq \budget \\
& & & X_{i} \in \{0,1\} \; i = 1, \ldots, m_{win}. & 
\end{aligned}
\end{equation}

\begin{proposition}
	Solving Eq.~\ref{eq:main2} is NP-hard.
\end{proposition}

\begin{proof}
	\todo{complete}
\end{proof}

Simultaneous auctions combining infinite supply with multiple buyers and sellers is far from easy \todo{elaborate}. Thus, in this paper we focus on an action-free market as described next. 

\subsection{An Auction-Free Market}

\todo{we begin with problem definition assuming auctions and than relax the auction such that each buyer is allocated to a seller and in case there is a match (the price the buyer is willing to pay is higher that the price limit set by the seller), the buyer gets the product}.

\todo{relax the problem, without $w$}

\subsection{Price Prediction and Bidding}

\todo{estimating costs}

\todo{biding strategies}
